\section{Conclusion}
\label{sec:conclusion}

This survey explores research on practical aspects of ITP systems to accumulate identified domain-specific challenges in large scale proof development as well as organizes proposed methodologies to tackle these challenges based on the proof engineering concerns they address. Efficiency and maintainability are two main challenges identified by researchers and proof engineers involved in past large-scale verification projects. Proofs in scaled development require to be adaptable i.e., robust and easily modifiable, well-organized, and reusable to promote efficient and maintainable proof development. Automation and portability of proofs among different ITP systems improve the performance of proof developers, thereby the efficiency of overall development. Proofs also need maintenance in the face of inevitable changes over time and incorporating these changes in large-scale development is another challenging task. Refactoring and repair are two automated techniques that researchers are actively working on to adapt or recover proofs automatically when change happens. Proof assistants are now being integrated with these techniques to provide better tooling support within the system. Many present ITP systems have Integrated Development Environments(IDE) too. Some of them provide project management systems and asynchronous development capability and all are evolving to better support the development process. 

Active research are now being carried out on most of these aspects mentioned above and they have lots of room for improvement. Proof reusability through type equivalence as well as proof refactoring and repair have the most promising prospect for future research work. Well-developed methods from software engineering can also be borrowed to advance proof engineering. However, proof development is inherently different from software development. Hence, comprehensive usability studies need to be performed to draw out specific requirements for this domain.





