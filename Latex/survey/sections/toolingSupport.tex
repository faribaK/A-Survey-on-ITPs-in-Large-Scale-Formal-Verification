\section{Tooling and IDE Support}
\label{sec:tooling}

Tooling and IDE support from proof assistants aid in proof development by providing means for automation, theorem searching, refactoring, and repairing within the system as well as by helping in project management through integrated development environments. Several of these tooling supports have already been discussed in previous sections. In section~\ref{sec:proofDev}, automated theorem proving tool Hammer, library mining tool Mizar Proof Advisor, proof reuse tool DEVOID have been covered. Section~\ref{sec:maintenance} includes a description of proof refactoring tools Levity, POLAR, Tactician, Chick, RefactorAgda, and proof repairing tool PUMPKIN PATCH. However, some tools also by default support data types and theorem search. For example, Coq has \emph{Search} command that can find relevant proofs form the standard library. Isabelle provides similar support through \emph{find\_consts} and \emph{find\_theorems} commands. These tooling supports can speed up the development process and improve maintainability to a great extent (see section~\ref{sec:proofDev} and~\ref{sec:maintenance} for detail). Some of these tools are not matured yet, though, and still have room for improvement. 

Proof assistants now alos have IDEs for development, which is a significant improvement over the initial REPL(Read Eval Print Loop) interface they had. CoqIDE and Isabelle/jEdit are the most popular IDEs for Coq and Isabelle. Proof general is another popular IDE within the Coq community, although it can be used with other proof assistants like Lego, Isabelle. It can also be extended with new functionalities like auto-completion. CoqIDE, along with another lesser-known IDE, Coqoon for Coq have support for project management. IDEs are also increasingly promoting asynchronous development. Previously, it was not possible to have a checker verifying some proofs when the user is modifying others. Now, it is supported in CoqIDE, Coqoon, as well as in Isabelle/jEdit. There is an interesting fact about Nuprl, however. While all most ITP systems followed the path from REPL through Emac to IDEs, Nuprl was launched with a graphical interface. Agda, however, does not have any IDE support. Proof editing in Agda is done through Emacs or Atom. In Emacs mode, as described in section~\ref{sec:overview}, the user can define holes in proof and have it type-checked and then fill in those holes later. 

Both tooling and IDE has come a long way for proof assistants and can be improved to better support large-scale development. There is a possibility that more productivity tools like refactoring and repairing would emerge in the coming years~\cite{Ringer_et_al_2019}. Another support that is scarce at this moment is debugging tools for proof assistants. Notably, the Tactic-based proofs do not allow any user interaction while running tactics or tacticals. If a tactic or tactical fails, the proof state goes back to the step before running them without providing any information on how it failed. It could be helpful to know the intermediate information for debugging purposes as well as for deciding the next move. There are some works in the tactic generalization that have the same idea and can be adapted for tooling as well~\cite{Felty_Howe_1994}. Large scale development also benefits from advanced IDEs. Tools like mentioned above can be integrated into IDE as plugins. The idea for useful features in development can be borrowed from software engineering and implemented in extensible IDEs like Proof General. Moreover, thorough usability analyses for proof development need to be performed to identify domain-specific requirements. 


%The extensibility of Proof General is one of the reasons for its over the years popularity. Other IDEs can also follow this direction. 