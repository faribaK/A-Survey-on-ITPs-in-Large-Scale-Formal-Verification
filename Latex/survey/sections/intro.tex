\section{Introduction}
\label{sec:intro}

In recent years, interactive theorem proving (ITP) that assists humans in the intriguing task of building formal machine-checked proofs, has been emerging in the formidable area of large-scale complex system formal verification, leading to new challenges in proof development and maintenance. In theorem proving, the goal of proofs is to validate the correctness of a mathematical statement. Any mathematical proof can be spelled out into simple steps that can be verified irrefutably, and this is the main idea behind the machine checking of proofs. Proofs need to be translated from informal language to formal language i.e. formal system, where each proof step is either an axiom of the formal system or can be inferred from one through its valid inference rules. However, in the interactive arrangement, the machine's role is not only to check whether a human written proof i.e., proof script, is valid in some formal system but, more importantly, to assist in finding that valid formal proof. Different formal systems have prompted the development of different interactive theorem proving systems, also known as proof assistants, who play this role of machine. However, formalization of a system is different from the formalization in mathematics in the sense that it entails more than just the aforementioned translation. One needs to identify the system's properties that ensure its correctness before going about translating them into formal statements and finding corresponding proofs. Intricacy and scale of the set of properties reflect the complexity and size of the underlying system, and specification of these properties often evolves with the system as well. With the increasing formal verification of the large-scale complex system in proof assistants following successful ventures like verified C compiler CompCert C~\cite{Leroy_2006}, proof development is now getting scaled at an unprecedented rate. Therefore, proof development too calls for methodologies to facilitate efficiency and maintainability i.e., proof engineering, just like efficient software development demands for software engineering. These proof engineering aspects of proof development in ITP systems are considered as practical aspects of proof assistants in this survey. Some impressive works have already been done by researchers to identify challenges in large-scale system verification along with efficient methodologies to handle them. Still, proof engineering is far away from reaching the maturity level of software engineering. This survey presents existing research on practical aspects of ITP systems in large-scale system verification by categorizing them with respect to the proof engineering concerns they address i.e., identifying the challenges and requirements in large-scale system verification (Section~\ref{sec:ITPinLarge}), providing methodologies for efficient development of maintainable proofs (Section~\ref{sec:proofDev}), aiding in maintenance of existing proofs over time through automated techniques (Section~\ref{sec:maintenance}), supporting development through tooling and Integrated Development Environment (IDE) from ITP systems (Section~\ref{sec:tooling}). However, before diving into research on practical aspects, this survey provides an overview of the ITP systems to acquaint readers with salient characteristics of different systems (Section~\ref{sec:overview}). Throughout the survey, it also points out the difference among ITP systems both in terms of their features and their practical aspects, as well as draws attention to significant challenges and open problems identified by the researchers.

%which is essential to understand the techniques presented in the literature
%Still, proof engineering is far away from reaching the maturity level of software engineering. In this survey, I am going to present existing research on practical aspects of ITP systems in large-scale development by categorizing them with respect to the proof engineering concerns they address i.e., i.e. identifying the challenges and requirements in large-scale development (Section ~\ref{sec:ITPinLarge}) and then, finding methodologies and  to tackle them. These methods and techniques across literature can further be categorized into two categories - a) proactive measures to be followed during proof creation to facilitate efficient development of maintainable proofs (Section ~\ref{sec:proofDev}), and b) reactive measures to be applied over time to adapt existing proofs to changes(Section ~\ref{sec:maintenance}). 

%Finding challenges and requirements for large scale development  These challenges and the requirements for large scale development will be discussed in section~\ref{sec:ITPinLarge}. Methodologies to tackle these challenges can further be categorized into proactive measures to be followed during proof design and development and reactive measures to   i.e. identifying the challenges and requirements in large-scale system verification (Section ~\ref{sec:ITPinLarge}), providing design methodologies for efficient development of maintainable proofs (Section ~\ref{sec:proofDev}), aiding in maintenance of proofs over time through automated techniques (Section ~\ref{sec:maintenance}), supporting development through tooling and IDE from ITP system (Section ~\ref{sec:tooling}). However, before diving into research on practical aspects, I will provide an overview of the ITP systems and their salient characteristics (Section ~\ref{sec:overview}), which is essential to understand the aforementioned techniques presented in the literature. This whole survey will also attempt to point out the difference among ITP systems both in terms of their features and their practical aspects, as well as will try draw attention to significant challenges and open problems identified by the researchers.

%Still, proof engineering is far away from reaching the maturity level of software engineering. In this survey, I am going to present existing research on practical aspects of ITP systems in large-scale development by categorizing them with respect to the proof engineering concerns they address. Identifying challenges and requirements for large scale development will be discussed in Section~\ref{sec:ITPinLarge}.    i.e. identifying the challenges and requirements in large-scale system verification (Section ~\ref{sec:ITPinLarge}), providing design methodologies for efficient development of maintainable proofs (Section ~\ref{sec:proofDev}), aiding in maintenance of proofs over time through automated techniques (Section ~\ref{sec:maintenance}), supporting development through tooling and IDE from ITP system (Section ~\ref{sec:tooling}). However, before diving into research on practical aspects, I will provide an overview of the ITP systems and their salient characteristics (Section ~\ref{sec:overview}), which is essential to understand the aforementioned techniques presented in the literature. This whole survey will also attempt to point out the difference among ITP systems both in terms of their features and their practical aspects, as well as will try draw attention to significant challenges and open problems identified by the researchers.

%through formalization
%in the readers need an introduction to the ITP systems and their characteristics to understand the techniques presented on the 
%Different interactive theorem proving systems, also known as proof assistants, has been developed with the emergence of different formal systems.
%Nevertheless, to reach the maturity level of software engineering, researchers need to find and address open challenges
%with the help of some proof assistant




